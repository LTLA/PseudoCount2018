\documentclass[10pt,letterpaper]{article}
\usepackage[top=0.85in,left=2.75in,footskip=0.75in,marginparwidth=2in]{geometry}

% use Unicode characters - try changing the option if you run into troubles with special characters (e.g. umlauts)
\usepackage[utf8]{inputenc}

% clean citations
\usepackage{cite}

% hyperref makes references clicky. use \url{www.example.com} or \href{www.example.com}{description} to add a clicky url
\usepackage{nameref,hyperref}

% line numbers
\usepackage[right]{lineno}

% improves typesetting in LaTeX
\usepackage{microtype}
\DisableLigatures[f]{encoding = *, family = * }

% text layout - change as needed
\raggedright
\setlength{\parindent}{0.5cm}
\textwidth 5.25in 
\textheight 8.75in

% Remove % for double line spacing
%\usepackage{setspace} 
%\doublespacing

% use adjustwidth environment to exceed text width (see examples in text)
\usepackage{changepage}

% adjust caption style
\usepackage[aboveskip=1pt,labelfont=bf,labelsep=period,singlelinecheck=off]{caption}

% remove brackets from references
\makeatletter
\renewcommand{\@biblabel}[1]{\quad#1.}
\makeatother

% headrule, footrule and page numbers
\usepackage{lastpage,fancyhdr,graphicx}
\usepackage{epstopdf}
\pagestyle{myheadings}
\pagestyle{fancy}
\fancyhf{}
\rfoot{\thepage/\pageref{LastPage}}
\renewcommand{\footrule}{\hrule height 2pt \vspace{2mm}}
\fancyheadoffset[L]{2.25in}
\fancyfootoffset[L]{2.25in}

% use \textcolor{color}{text} for colored text (e.g. highlight to-do areas)
\usepackage{color}

% define custom colors (this one is for figure captions)
\definecolor{Gray}{gray}{.25}

% this is required to include graphics
\usepackage{graphicx}

% use if you want to put caption to the side of the figure - see example in text
\usepackage{sidecap}

% use for have text wrap around figures
\usepackage{wrapfig}
\usepackage[pscoord]{eso-pic}
\usepackage[fulladjust]{marginnote}
\reversemarginpar

% Adding multirow.
\usepackage{multirow}

% Other required things:
\usepackage{color}
\usepackage{subcaption}
\captionsetup[subfigure]{justification=centering}
\usepackage{amsmath}

% document begins here
\begin{document}
\vspace*{0.35in}

% title goes here:
\begin{flushleft}
{\Large
    \textbf\newline{Using an appropriate pseudo-count for log-transformation of normalized single-cell RNA sequencing data}
}
\newline

% authors go here:
%\\
Aaron Lun\textsuperscript{1,*}
\\
\bigskip
\bf{1} Cancer Research UK Cambridge Institute, University of Cambridge, Li Ka Shing Centre, Robinson Way, Cambridge CB2 0RE, United Kingdom \\
\bigskip

\end{flushleft}

\section{Background}
Log-transformed expression values are widely used in analyses of single-cell RNA sequencing (scRNA-seq) and other transcriptomic data.
This is driven the simplicity of the log-transformation and its moderate variance stabilizing capabilities on many types of non-negative data.
In particular, the log-transformation reduces the impact of stochastic fluctuations in the counts for high-abundance genes.
This would otherwise result in large differences in the counts that are mostly uninteresting as the fold changes are small.
Differences between log-values are also approximately interpretable as log-fold changes between cells, which are often more relevant than differences in the counts.
This is useful for ensuring that the relative rather than absolute differences in the counts are used in distance-based procedures like clustering or trajectory construction.

Despite its popularity, the log-transformation has a number of problems such as incomplete variance stabilization and arbitrariness in the choice of pseudo-count.
One issue of particular interest is that the mean of the log-counts is not generally the same as the log-mean count \cite{hicks2017missing}.
This is problematic in (sc)RNA-seq contexts where the log-transformation is applied to normalized expression data.
Here, normalization is performed to remove inter-sample biases on the count scale \cite{robinson2010scaling,lun2016pooling}.
For genes that are not differentially expressed (DE), the expectation of the normalized expression is the same between samples.
However, the expectation of the log-normalized values may not be the same, resulting in spurious differences between samples/cells in the log-scale.

In this report, we describe the nature and impact of the discrepancy between log-mean and mean-log values.
We also show how the added pseudo-count can be increased to cap the discrepancy at the cost of reducing interpretability.

\section{Computing a suitable pseudo-count}

\subsection{Derivation}

Let $X_i$ denote a random variable for a non-negative count of a gene $g$ in cell $i$.
The log-transformed normalized expression value is defined as 
\[
Z_i = \log(X_i / s_i + c) \;,
\]
where $s_i$ is the size factor for $i$ and $c$ is the pseudo-count.
(We will omit the gene indexing for clarity, as we are only referring to a single gene throughout.)
The second-order Taylor series approximation for the expectation of $Z_i$ is
\[
E(Z_i) = \log[E(X_i/s_i) + c] - \frac{\mbox{var}(X_i)s_i^{-2}}{2[E(X_i/s_i) + c]^2} \;.
\]
If we further assume that $X_i$ follows a negative binomial (NB) distribution with mean $s_i\mu$ and dispersion $\varphi$, we obtain
\[
E(Z_i) =  \log[\mu + c] - \frac{\mu s_i^{-1} + \varphi \mu^2}{2(\mu + c)^2} \;.
\]

Now, consider two cells $i=1$ and $i=2$ that differ only in their $s_i$.
The difference in $E(Z_i)$ represents the expected log-fold change between cells, which is
\begin{align*}
\Delta_{12} 
&= \frac{\mu s_1^{-1} + \varphi \mu^2}{2(\mu + c)^2} - \frac{\mu s_2^{-1} + \varphi \mu^2}{2(\mu + c)^2} \\ 
&= \frac{\mu (s_1^{-1} - s_2^{-1})}{2(\mu + c)^2} \;.
\end{align*}
$\Delta_{12}$ is a spurious difference as the log-fold change between cells should be zero after normalization.
This error in the log-fold change is maximized when $\mu = c$, yielding 
\[
\tilde\Delta_{12} = \frac{(s_1^{-1} - s_2^{-1})}{8c} \;.
\]
Thus, we can cap the maximum error $\tilde\Delta_{12}$ regardless of the values of $s_i$ by defining
\[
    c \propto |s_1^{-1} - s_2^{-1}| \;.
\]


\bibliography{ref}
\bibliographystyle{unsrt}

\end{document}
