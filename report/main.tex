\documentclass[10pt,letterpaper]{article}
\usepackage[top=0.85in,left=2.75in,footskip=0.75in,marginparwidth=2in]{geometry}

% use Unicode characters - try changing the option if you run into troubles with special characters (e.g. umlauts)
\usepackage[utf8]{inputenc}

% clean citations
\usepackage{cite}

% hyperref makes references clicky. use \url{www.example.com} or \href{www.example.com}{description} to add a clicky url
\usepackage{nameref,hyperref}

% line numbers
\usepackage[right]{lineno}

% improves typesetting in LaTeX
\usepackage{microtype}
\DisableLigatures[f]{encoding = *, family = * }

% text layout - change as needed
\raggedright
\setlength{\parindent}{0.5cm}
\textwidth 5.25in 
\textheight 8.75in

% Remove % for double line spacing
%\usepackage{setspace} 
%\doublespacing

% use adjustwidth environment to exceed text width (see examples in text)
\usepackage{changepage}

% adjust caption style
\usepackage[aboveskip=1pt,labelfont=bf,labelsep=period,singlelinecheck=off]{caption}

% remove brackets from references
\makeatletter
\renewcommand{\@biblabel}[1]{\quad#1.}
\makeatother

% headrule, footrule and page numbers
\usepackage{lastpage,fancyhdr,graphicx}
\usepackage{epstopdf}
\pagestyle{myheadings}
\pagestyle{fancy}
\fancyhf{}
\rfoot{\thepage/\pageref{LastPage}}
\renewcommand{\footrule}{\hrule height 2pt \vspace{2mm}}
\fancyheadoffset[L]{2.25in}
\fancyfootoffset[L]{2.25in}

% use \textcolor{color}{text} for colored text (e.g. highlight to-do areas)
\usepackage{color}

% define custom colors (this one is for figure captions)
\definecolor{Gray}{gray}{.25}

% this is required to include graphics
\usepackage{graphicx}

% use if you want to put caption to the side of the figure - see example in text
\usepackage{sidecap}

% use for have text wrap around figures
\usepackage{wrapfig}
\usepackage[pscoord]{eso-pic}
\usepackage[fulladjust]{marginnote}
\reversemarginpar

% Adding multirow.
\usepackage{multirow}

% Other required things:
\usepackage{color}
\usepackage{subcaption}
\captionsetup[subfigure]{justification=centering}
\usepackage{amsmath}

% document begins here
\begin{document}
\vspace*{0.35in}

% title goes here:
\begin{flushleft}
{\Large
    \textbf\newline{Using an appropriate pseudo-count for log-transformation of normalized single-cell RNA sequencing data}
}
\newline

% authors go here:
%\\
Aaron Lun\textsuperscript{1,*}
\\
\bigskip
\bf{1} Cancer Research UK Cambridge Institute, University of Cambridge, Li Ka Shing Centre, Robinson Way, Cambridge CB2 0RE, United Kingdom \\
\bigskip

\end{flushleft}

\section{Computing a suitable pseudo-count}

\subsection{Derivation}

Let $X_i$ denote a random variable for a non-negative count of a gene $g$ in cell $i$.
The log-transformed normalized expression value is defined as 
\[
Z_i = \log(X_i / s_i + c) \;,
\]
where $s_i$ is the size factor for $i$ and $c$ is the pseudo-count.
(We will omit the gene indexing for clarity, as we are only referring to a single gene throughout.)
The second-order Taylor series approximation for the expectation of $Z_i$ is
\[
E(Z_i) = \log[E(X_i/s_i) + c] - \frac{\mbox{var}(X_i)s_i^{-2}}{2[E(X_i/s_i) + c]^2} \;.
\]
If we further assume that $X_i$ follows a negative binomial (NB) distribution with mean $s_i\mu$ and dispersion $\varphi$, we obtain
\[
E(Z_i) =  \log[\mu + c] - \frac{\mu s_i^{-1} + \varphi \mu^2}{2(\mu + c)^2} \;.
\]

Now, consider two cells $i=1$ and $i=2$ that differ only in their $s_i$.
The difference in $E(Z_i)$ represents the expected log-fold change between cells, which is
\begin{align*}
\Delta_{12} 
&= \frac{\mu s_1^{-1} + \varphi \mu^2}{2(\mu + c)^2} - \frac{\mu s_2^{-1} + \varphi \mu^2}{2(\mu + c)^2} \\ 
&= \frac{\mu (s_1^{-1} - s_2^{-1})}{2(\mu + c)^2} \;.
\end{align*}
$\Delta_{12}$ is a spurious difference as the log-fold change between cells should be zero after normalization.
This error in the log-fold change is maximized when $\mu = c$, yielding 
\[
\tilde\Delta_{12} = \frac{(s_1^{-1} - s_2^{-1})}{8c} \;.
\]
Thus, we can cap the maximum error $\tilde\Delta_{12}$ regardless of the values of $s_i$ by defining
\[
    c \propto |s_1^{-1} - s_2^{-1}| \;.
\]


\bibliography{ref}
\bibliographystyle{unsrt}

\end{document}
